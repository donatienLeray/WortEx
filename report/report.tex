\documentclass{article}[12pt]
\usepackage{silence}

\hbadness=9999999

\newcommand{\aetitle}{Wortex} % Title of the report
\newcommand{\studentOne}{Artur Schäfer} % Name 1
\newcommand{\studentTwo}{Donatien Leray} % Name 2

\begin{document}

    \twocolumn[{\begin{small}
\begin{minipage}{0.49 \linewidth}
  Linguistic gaming with Python\\
  WS 23/24
\end{minipage}
\begin{minipage}{0.5\linewidth}
  \begin{flushright}
    \studentOne\\
    \studentTwo
  \end{flushright}
\end{minipage}
\end{small}}
{\begin{center}
\begin{sffamily}\Huge\bfseries \aetitle \end{sffamily}
\end{center}}
\vskip 3em]

    
    \onecolumn

    \section*{Introduction}

    Leaning new words can be pretty boring, so we have created a game to make
    it more exciting. Wortex is a linguistic word game. It is designed to
    challenge players' linguistic skills by letting them write all the words
    they can form out of a given set of letters. In this report, we will
    discuss the objectives, mechanics, and features of the game. Linguistic
    games offer a fun way to improve vocabulary and spelling skills. Wortex is
    implemented in Python and uses the Pygame library for its graphical user
    interface.

    \section*{Objectives}

    The primary objective of Wortex is to form as many words as possible from a
    given set of seven letters. Players must create words of at least three
    letters, up  to the maximum length of seven. \\\\ The more words a player
    forms, the higher their score. To achieve a top score, players should aim
    to construct not only words, that are as long as possible but also rare.
    The game ends if the timer runs out or if the player finds all possible
    words, that can be formed from the set of given letters.

    \section*{Mechanics}

    \subsection*{Menu}

    Upon launching the game, players are greeted with a menu. Here, they can
    choose their preferred language (English or German) and view a scoreboard
    showcasing the highest scores achieved.

    Upon launching Wortex, players are presented with a menu. Here, they can
    choose a language (English or German) they prefer to play with. There is also
    a button to view a scoreboard, which is showcasing the highest scores ever
    achieved.

    \subsection*{Gameplay}

    After selecting a language and the play button is pressed, the game begins.
    It provides the player with a set of seven letters. On the screen you can
    see a timer, the score, and a list of words that appears if the player
    types a correct word. In the middle there is a circle that represents the time and
    in that circle there are the letters that can be typed to form words. Also
    there is information about how many words can be found in the given set.
    The player can form the words by typing the letters on the keyboard. With
    the escape key, the player can reset his input or by deleting the last
    letter with backspace. After guessing a correct word, the input gets
    cleared and the found word is added to the list of found words. If a seven
    letter word is found, the player gets bonus time to find more words. Also
    if the player guessed a word that is rare, the player gets more points than
    for a common word. There are no penalties for guessing wrong. The player is
    just wasting time that could have been used to find more words.

    \subsection*{Features}
    
    After the timer runs out or the player achieves to find all possible words,
    the game ends and the player is presented with a screen that shows the
    score and all the words that could be found and also were found by the
    player marked as green. The words can now be clicked on and it opens the
    browser to the definition of the word. In German we open the online Duden
    dictionary and for English the online Oxford dictionary. The player can
    then repeat the process by pressing the play again button or go back to the
    main menu. From the main menu the player can view a leaderboard, which
    shows the highest scores ever achieved. If the game is started again, the
    player gets a new set of seven letters to guess words from.

\end{document}
